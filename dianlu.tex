% 若编译失败,且生成 .synctex(busy) 辅助文件,可能有两个原因:
% 1. 需要插入的图片不存在:Ctrl + F 搜索 'figure' 将这些代码注释/删除掉即可
% 2. 路径/文件名含中文或空格:更改路径/文件名即可

% ------------------------------------------------------------- %
% >> ------------------ 文章宏包及相关设置 ------------------ << %
% 设定文章类型与编码格式
\documentclass[UTF8]{report}		

% 本文特殊宏包
    \usepackage{siunitx} % 埃米单位

% 本文的特殊宏定义
\def\Im{\mathrm{\,Im\,}}
\def\Re{\mathrm{\,Re\,}}
\def\Ln{\mathrm{\,Ln\,}}
\def\Arg{\mathrm{\,Arg\,}}
\def\Arccos{\mathrm{\,Arccos\,}}
\def\Arcsin{\mathrm{\,Arcsin\,}}
\def\Arctan{\mathrm{\,Arctan\,}}

% 通用宏定义
\def\N{\mathbb{N}}
\def\F{\mathbb{F}}
\def\Z{\mathbb{Z}}
\def\Q{\mathbb{Q}}
\def\R{\mathbb{R}}
\def\C{\mathbb{C}}
\def\T{\mathbb{T}}
\def\S{\mathbb{S}}
\def\A{\mathbb{A}}
\def\I{\mathscr{I}}
\def\d{\mathrm{d}}
\def\p{\partial}


% 导入基本宏包
    \usepackage[UTF8]{ctex}     % 设置文档为中文语言
    \usepackage[colorlinks, linkcolor=blue, anchorcolor=blue, citecolor=blue, urlcolor=blue]{hyperref}  % 宏包:自动生成超链接 (此宏包与标题中的数学环境冲突)
    % \usepackage{docmute}    % 宏包:子文件导入时自动去除导言区,用于主/子文件的写作方式,\include{./51单片机笔记}即可。注:启用此宏包会导致.tex文件capacity受限。
    \usepackage{amsmath}    % 宏包:数学公式
    \usepackage{mathrsfs}   % 宏包:提供更多数学符号
    \usepackage{amssymb}    % 宏包:提供更多数学符号
    \usepackage{pifont}     % 宏包:提供了特殊符号和字体
    \usepackage{extarrows}  % 宏包:更多箭头符号
    \usepackage{multicol}   % 宏包:支持多栏 
    \usepackage{graphicx}   % 宏包:插入图片
    \usepackage{float}      % 宏包:设置图片浮动位置
    %\usepackage{article}    % 宏包:使文本排版更加优美
    \usepackage{tikz}       % 宏包:绘图工具
    %\usepackage{pgfplots}   % 宏包:绘图工具
    \usepackage{enumerate}  % 宏包:列表环境设置
    \usepackage{enumitem}   % 宏包:列表环境设置

% 文章页面margin设置
    \usepackage[a4paper]{geometry}
        \geometry{top=1in}  % 1 inch= 2.46 cm, 0.75 inch = 1.85 cm
        \geometry{bottom=1in}
        \geometry{left=0.75in}
        \geometry{right=0.75in}   % 设置上下左右页边距
        \geometry{marginparwidth=1.75cm}    % 设置边注距离(注释、标记等)

% 配置数学环境
    \usepackage{amsthm} % 宏包:数学环境配置
    % theorem-line 环境自定义
        \newtheoremstyle{MyLineTheoremStyle}% <name>
            {11pt}% <space above>
            {11pt}% <space below>
            {}% <body font> 使用默认正文字体
            {}% <indent amount>
            {\bfseries}% <theorem head font> 设置标题项为加粗
            {:}% <punctuation after theorem head>
            {.5em}% <space after theorem head>
            {\textbf{#1}\thmnumber{#2}\ \ (\,\textbf{#3}\,)}% 设置标题内容顺序
        \theoremstyle{MyLineTheoremStyle} % 应用自定义的定理样式
        \newtheorem{LineTheorem}{Theorem.\,}
    % theorem-block 环境自定义
        \newtheoremstyle{MyBlockTheoremStyle}% <name>
            {11pt}% <space above>
            {11pt}% <space below>
            {}% <body font> 使用默认正文字体
            {}% <indent amount>
            {\bfseries}% <theorem head font> 设置标题项为加粗
            {:\\ \indent}% <punctuation after theorem head>
            {.5em}% <space after theorem head>
            {\textbf{#1}\thmnumber{#2}\ \ (\,\textbf{#3}\,)}% 设置标题内容顺序
        \theoremstyle{MyBlockTheoremStyle} % 应用自定义的定理样式
        \newtheorem{BlockTheorem}[LineTheorem]{Theorem.\,} % 使用 LineTheorem 的计数器
    % definition 环境自定义
        \newtheoremstyle{MySubsubsectionStyle}% <name>
            {11pt}% <space above>
            {11pt}% <space below>
            {}% <body font> 使用默认正文字体
            {}% <indent amount>
            {\bfseries}% <theorem head font> 设置标题项为加粗
            { \indent}% <punctuation after theorem head>
            {0pt}% <space after theorem head>
            {\textbf{#3}}% 设置标题内容顺序
        \theoremstyle{MySubsubsectionStyle} % 应用自定义的定理样式
        \newtheorem{definition}{}

%宏包:有色文本框(proof环境)及其设置
    \usepackage[dvipsnames,svgnames]{xcolor}    %设置插入的文本框颜色
    \usepackage[strict]{changepage}     % 提供一个 adjustwidth 环境
    \usepackage{framed}     % 实现方框效果
        \definecolor{graybox_color}{rgb}{0.95,0.95,0.96} % 文本框颜色。修改此行中的 rgb 数值即可改变方框纹颜色,具体颜色的rgb数值可以在网站https://colordrop.io/ 中获得。(截止目前的尝试还没有成功过,感觉单位不一样)(找到喜欢的颜色,点击下方的小眼睛,找到rgb值,复制修改即可)
        \newenvironment{graybox}{%
        \def\FrameCommand{%
        \hspace{1pt}%
        {\color{gray}\small \vrule width 2pt}%
        {\color{graybox_color}\vrule width 4pt}%
        \colorbox{graybox_color}%
        }%
        \MakeFramed{\advance\hsize-\width\FrameRestore}%
        \noindent\hspace{-4.55pt}% disable indenting first paragraph
        \begin{adjustwidth}{}{7pt}%
        \vspace{2pt}\vspace{2pt}%
        }
        {%
        \vspace{2pt}\end{adjustwidth}\endMakeFramed%
        }

% 外源代码插入设置
    % matlab 代码插入设置
    %\usepackage{matlab-prettifier}
    %    \lstset{
    %        style=Matlab-editor,  % 继承matlab代码颜色等
    %    }
    %\usepackage[most]{tcolorbox} % 引入tcolorbox包 
    %\usepackage{listings} % 引入listings包
    %    \tcbuselibrary{listings, skins, breakable}
    %    \newfontfamily\codefont{Consolas} % 定义需要的 codefont 字体
    %    \lstdefinestyle{matlabstyle}{
    %        language=Matlab,
    %        basicstyle=\small\ttfamily\codefont,    % ttfamily 确保等宽 
    %        breakatwhitespace=false,
    %        breaklines=true,
    %        captionpos=b,
    %        keepspaces=true,
    %        numbers=left,
    %        numbersep=15pt,
    %        showspaces=false,
    %        showstringspaces=false,
    %        showtabs=false,
    %        tabsize=2
    %    }
    %    \newtcblisting{matlablisting}{
    %        arc=2pt,        % 圆角半径
    %        top=-5pt,
    %        bottom=-5pt,
    %        left=1mm,
    %        listing only,
    %        listing style=matlabstyle,
    %        breakable,
    %        colback=white   % 选一个合适的颜色
    %    }
% table 支持
    \usepackage{booktabs}   % 宏包:三线表
    \usepackage{tabularray} % 宏包:表格排版
    \usepackage{longtable}  % 宏包:长表格


%figure 设置
%    \usepackage{graphicx}  % 支持 jpg, png, eps, pdf 图片 
%    \usepackage{svg}       % 支持 svg 图片
%        \svgsetup{
%             指向 inkscape.exe 的路径
%            inkscapeexe = C:/aa_MySame/inkscape/bin/inkscape.exe, 
%            inkscapeexe = C:/aa_MySame/inkscape/bin/inkscape.exe, 
%             一定程度上修复导入后图片文字溢出几何图形的问题
%            inkscapelatex = false                 
%        }
%    \usepackage{subcaption} % subfigure 子图支持

%图表进阶设置
%    \usepackage{caption}    % 图注、表注
%        \captionsetup[figure]{name=图}  
%        \captionsetup[table]{name=表}
%        \captionsetup{labelfont=bf, font=small}
%    \usepackage{float}     % 图表位置浮动设置 

% 圆圈序号自定义
    \newcommand*\circled[1]{\tikz[baseline=(char.base)]{\node[shape=circle,draw,inner sep=0.8pt, line width = 0.03em] (char) {\small \bfseries #1};}}   % TikZ solution

% 列表设置
%    \usepackage{enumitem}   % 宏包:列表环境设置
%        \setlist[enumerate]{itemsep=0pt, parsep=0pt, topsep=0pt, partopsep=0pt, leftmargin=3.5em} 
%        \setlist[itemize]{itemsep=0pt, parsep=0pt, topsep=0pt, partopsep=0pt, leftmargin=3.5em}
%        \newlist{circledenum}{enumerate}{1} % 创建一个新的枚举环境  
%        \setlist[circledenum,1]{  
%            label=\protect\circled{\arabic*}, % 使用 \arabic* 来获取当前枚举计数器的值,并用 \circled 包装它  
%            ref=\arabic*, % 如果需要引用列表项,这将决定引用格式(这里仍然使用数字)
%            itemsep=0pt, parsep=0pt, topsep=0pt, partopsep=0pt, leftmargin=3.5em
%        }  

% 其它设置
    % 脚注设置
        \renewcommand\thefootnote{\ding{\numexpr171+\value{footnote}}}
    % 参考文献引用设置
        \bibliographystyle{unsrt}   % 设置参考文献引用格式为unsrt
        \newcommand{\upcite}[1]{\textsuperscript{\cite{#1}}}     % 自定义上角标式引用
    % 文章序言设置
        \newcommand{\cnabstractname}{序言}
        \newenvironment{cnabstract}{%
            \par\Large
            \noindent\mbox{}\hfill{\bfseries \cnabstractname}\hfill\mbox{}\par
            \vskip 2.5ex
            }{\par\vskip 2.5ex}

% 文章默认字体设置
    \usepackage{fontspec}   % 宏包:字体设置
        \setmainfont{SimSun}    % 设置中文字体为宋体字体
        \setCJKmainfont[AutoFakeBold=3]{SimSun} % 设置加粗字体为 SimSun 族,AutoFakeBold 可以调整字体粗细
        \setmainfont{Times New Roman} % 设置英文字体为Times New Roman

% 各级标题自定义设置
    \usepackage{titlesec}   
        \titleformat{\chapter}[hang]{\normalfont\huge\bfseries\centering}{第\,\thechapter\,章}{20pt}{}
        \titlespacing*{\chapter}{0pt}{-20pt}{20pt} % 控制上方空白的大小
        % section标题自定义设置 
        \titleformat{\section}[hang]{\normalfont\Large\bfseries}{§\,\thesection\,}{8pt}{}
        % subsubsection标题自定义设置
        %\titleformat{\subsubsection}[hang]{\normalfont\bfseries}{}{8pt}{}

% >> ------------------ 文章宏包及相关设置 ------------------ << %
% ------------------------------------------------------------- %

% ----------------------------------------------------------- %
% >> --------------------- 文章信息区 --------------------- << %
% 页眉页脚设置
    \usepackage{fancyhdr}   %宏包:页眉页脚设置
        \pagestyle{fancy}
        \fancyhf{}
        \cfoot{\thepage}
        \renewcommand\headrulewidth{1pt}
        \renewcommand\footrulewidth{0pt}
        \lhead{2024.8 -- 2025.1} 
        \chead{yinchao050313@gmail.com}    
        \rhead{yinchao23@mails.ucas.ac.cn}
%文档信息设置
    \title{电路分析基础}
    \author{尹超\\ \footnotesize 中国科学院大学,北京 100049\\ Carter Yin \\ \footnotesize University of Chinese Academy of Sciences, Beijing 100049, China}
    \date{\footnotesize 2024.8 -- 2025.1}
% >> --------------------- 文章信息区 --------------------- << %
% ----------------------------------------------------------- %

% 开始编辑文章

\begin{document} 
\zihao{5}             % 设置全文字号大小, -4 为小四, 5 为五号

% --------------------------------------------------------------- %
% >> --------------------- 封面序言与目录 --------------------- << %
% 封面
    \maketitle\newpage  
    \pagenumbering{Roman} % 页码为大写罗马数字
    \thispagestyle{fancy}   % 显示页码、页眉等

% 序言
    \begin{cnabstract}\normalsize 
        本笔记为电路分析基础的笔记整理\par
        讲课教师:王丹力\par
        助教:高玄歌\par
\end{cnabstract}    
\addcontentsline{toc}{chapter}{序言} % 手动添加为目录

% 目录
    \setcounter{tocdepth}{4}                % 目录深度(为1时显示到section)
    \tableofcontents                        % 目录页
    \addcontentsline{toc}{chapter}{目录}    % 手动添加此页为目录
    \thispagestyle{fancy}                   % 显示页码、页眉等 

% 收尾工作
    \newpage    
    \pagenumbering{arabic} 


    
% >> --------------------- 封面序言与目录 --------------------- << %
% --------------------------------------------------------------- %

\chapter{电路模型和电路定律}

\section{电路和电路模型}

\begin{definition}
    \textbf{5种基本的理想电路元件}
    \begin{itemize}
        \item 电阻元件:表示消耗电能的元件
        \item 电感元件:表示产生磁场,储存磁场能量的元件
        \item 电容元件:表示产生电场,储存电场能量的元件
        \item 电压源和电流源:表示将其它形式的能量转变成电能的元件
    \end{itemize}
    \textbf{注意}
    \begin{itemize}
        \item 5种基本理想电路元件有三个特征:
        \begin{enumerate}[label=(\alph*)]
            \item 只有两个端子
            \item 可以用电压、电流按数学方式描述
            \item 不能被分解为其他元件
        \end{enumerate}
    \end{itemize}
\end{definition}

\section{电流电压的参考方向}

\begin{definition}
    \textbf{基础概念}
    \begin{itemize}
        \item 电路中主要物理量
        \begin{itemize}
            \item 电流、电压、电荷、电功率、能量、磁通链等
            \item 物理量的表达是电流($I$)、电压($U$)、电荷($Q$)、电功率($P$)、能量($W$)、磁通链($\Psi$)
            \item 电路中电流、电压、功率等变量一般用小写字母表示 $i$,$u$,$p$
        \end{itemize}
        \item 线性电路分析主要关注的是电流、电压和功率
        \begin{itemize}
            \item 线性元件:参数与电压、电流无关的元件
        \end{itemize}
        \item 线性电路(linear circuit)
        \begin{itemize}
            \item 若描述电路特性的所有方程都是线性代数方程或微积分方程,则称这类电路是线性电路,否则为非线性电路。
            \item 线性电路:由线性元件、独立源、线性受控源组成的电路。
        \end{itemize}
    \end{itemize}

    \textbf{电流的参考方向}
    \begin{enumerate}
        \item 用箭头表示:箭头的指向为电流的参考方向
        \item 用双下标表示:如 $i_{AB}$,电流的参考方向由A指向B。
    \end{enumerate}

    \textbf{电压的参考方向}
    \begin{itemize}
        \item 实际电压方向:电位真正降低的方向。
        \item 电压参考方向的三种表示方式:
        \begin{enumerate}
            \item 用箭头表示:箭头的指向为电压的参考方向
            \item 用双下标表示:如 $u_{AB}$,电压的参考方向由A指向B。
            \item 用正负极性表示
        \end{enumerate}
    \end{itemize}

    \textbf{关联参考方向}
    \begin{itemize}
        \item 元件或支路的 $u$,$i$ 采用相同的参考方向称之为关联参考方向。反之,称为非关联参考方向。
    \end{itemize}
\end{definition}

\section{电功率和能量}

\begin{definition}
    \textbf{电功率}
    \begin{itemize}
        \item 概念:电功率是单位时间内电场力所做的功
    \end{itemize}
    \textbf{能量}
    \begin{itemize}
        \item 概念:从 $t_0$ 到 $t$ 元件吸收的能量
    \end{itemize}

    \textbf{电路吸收或发出功率的判断}
    \begin{itemize}
        \item $u$,$i$ 取关联参考方向
        \begin{itemize}
            \item $P = ui$ 表示元件吸收的功率
            \item $P > 0$ 吸收正功率 (实际吸收)
            \item $P < 0$ 吸收负功率 (实际发出)
        \end{itemize}
        \item $u$,$i$ 取非关联参考方向
        \begin{itemize}
            \item $P = ui$ 表示元件发出的功率
            \item $P > 0$ 发出正功率 (实际发出)
            \item $P < 0$ 发出负功率 (实际吸收)
        \end{itemize}
    \end{itemize}
\end{definition}

\section{电路元件}

\begin{definition}
    \textbf{电路元件} 是电路中最基本的组成单元。
    \begin{itemize}
        \item 5种基本的理想电路元件
        \begin{itemize}
            \item 电阻元件:表示消耗电能的元件
            \item 电感元件:表示产生磁场,储存磁场能量的元件
            \item 电容元件:表示产生电场,储存电场能量的元件
            \item 电压源和电流源:表示将其它形式的能量转变成电能的元件。
            \item 如果表征元件特性的数学关系式是线性关系,该元件称为线性元件,否则称为非线性元件。
        \end{itemize}
        \item 集总参数电路
        \begin{itemize}
            \item 由集总元件构成的电路
            \item 集总元件发生的电磁过程都集中在元件内部进行。
            \item 集总参数电路
            \begin{itemize}
                \item 当电路的几何尺寸远远小于电路中信号的波长时,电路周围的辐射效应很小,可以忽略不计,这时电流的能量完全分布在金属导线和电路元件之内,这样的电路成为集总参数电路。可以认为传送到电路各处的电磁能量同时到达,整个电路为电磁空间的一个点。这类电路就可以用电路模型来描述。
                \item 集总参数电路中 $u$、$i$ 可以是时间的函数,但与空间坐标无关。因此,任何时刻,流入两端元件一个端子的电流等于从另一端子流出的电流;端子间的电压为单值量。
            \end{itemize}
        \end{itemize}
    \end{itemize}
\end{definition}

\section{电阻元件}

\begin{definition}
    \textbf{电阻元件的定义} 对电流呈现阻力、消耗能量的元件。其特性可用 $u$~$i$ 平面上的一条曲线来描述:
    \[
    f(u, i) = 0
    \]
    \textbf{线性时不变电阻元件}
    \begin{itemize}
        \item $u$~$i$ 关系满足欧姆定律
        \item 伏安特性为一条过原点的直线
    \end{itemize}
    \textbf{功率和能量}
    \begin{itemize}
        \item 公式和参考方向必须配套使用!
    \end{itemize}
    \textbf{电阻的开路与短路}
    \begin{itemize}
        \item 开路
        \[
        i = 0, \quad u \neq 0, \quad R = \infty, \quad G = 0
        \]
        \item 短路
        \[
        u = 0, \quad i \neq 0, \quad R = 0, \quad G = \infty
        \]
    \end{itemize}
\end{definition}

\section{电压源和电流源}

\begin{definition}
    1. 理想电压源
    \textbf{定义}
    \begin{itemize}
        \item 电压源两端电压总能保持定值或一定的时间函数,其值与流过它的电流 $i$ 无关的元件叫理想电压源。
    \end{itemize}
    \textbf{电路符号}
    \begin{figure}[ht]
        \centering
        \includegraphics[width=0.5\textwidth]{dianya.png}
    \end{figure}
    \textbf{理想电压源的电压、电流关系}
    \begin{itemize}
        \item 电压源两端电压由电源本身决定,与外电路无关;与流经它的电流方向、大小无关。
        \item 通过电压源的电流由电压源及外电路共同决定。
        \item 电压源不能短路!
    \end{itemize}

    2. 理想电流源
    \textbf{定义}
    \begin{itemize}
        \item 理想电流源的输出电流总能保持定值或一定的时间函数,其值与它的两端电压 $u$ 无关的元件叫理想电流源。
    \end{itemize}
    \textbf{电路符号}
    \begin{figure}[ht]
        \centering
        \includegraphics[width=0.5\textwidth]{dianliu.png}
    \end{figure}
    \textbf{理想电流源的电压、电流关系}
    \begin{itemize}
        \item 电流源的输出电流由电源本身决定,与外电路无关;与它两端电压方向、大小无关。
        \item 电流源两端的电压由电流源及外电路共同决定。
        \item 电流源不能开路!
    \end{itemize}
\end{definition}

\section{基尔霍夫定律}

\begin{definition}
    \textbf{基尔霍夫定律包括:}
    \begin{itemize}
        \item 基尔霍夫电流定律(Kirchhoff’s Current Law, KCL)
        \item 基尔霍夫电压定律(Kirchhoff’s Voltage Law, KVL)
    \end{itemize}
    \textbf{基尔霍夫定律反映了电路中所有支路电压和电流所遵循的基本规律,是分析集总参数电路的基本定律;}
    \textbf{基尔霍夫定律(KCL、KVL)与元件特性(Voltage Current Relation, VCR)构成了电路分析的基础。}

    1. 相关名词
    \begin{itemize}
        \item 支路
        \begin{itemize}
            \item 电路中每一个两端元件就叫一条支路
            \item 电路中通过同一电流的分支
        \end{itemize}
        \item 节点
        \begin{itemize}
            \item 元件的连接点称为结点
            \item 三条以上支路的连接点称为结点
        \end{itemize}
        \item 路径
        \begin{itemize}
            \item 两结点间的一条通路,由支路构成
        \end{itemize}
        \item 回路
        \begin{itemize}
            \item 由支路组成的闭合路径
        \end{itemize}
        \item 网孔
        \begin{itemize}
            \item 对平面电路,其内部不含任何支路的回路称网孔。
            \item 网孔是回路,但回路不一定是网孔。
        \end{itemize}
    \end{itemize}

    2. 基尔霍夫电流定律(KCL)
    \begin{itemize}
        \item 在集总参数电路中,任意时刻,对任意结点流出(或流入)该结点电流的代数和等于零。
    \end{itemize}

    3. 基尔霍夫电压定律(KVL)
    \begin{itemize}
        \item 在集总参数电路中,任一时刻,沿任一回路,所有支路电压的代数和恒等于零。
        \item KVL也适用于电路中任一假想的回路。
    \end{itemize}
    \begin{figure}[ht]
        \centering
        \includegraphics[width=0.5\textwidth]{kvl.png}
    \end{figure}
\end{definition}


\chapter{电阻电路的等效变换}

\section{电路的等效变换}

\begin{definition}
    1.两端电路(网络)
    任何一个复杂的电路, 向外引出两个端子,且从一个端子流入的
电流等于从另一端子流出的电流,则称这一电路为二端网络 (或一
端口网络)。

2.两端电路等效的概念
两个两端电路,端口具有相同的电压、电流关系,则称它们是等效的
电路(对外等效)。
\end{definition}

\section{电阻的串联和并联}

\section{电阻的Y形联接和$\Delta$形联接的等效变换}

\begin{definition}

    \begin{figure}[H]
        \centering
        \includegraphics[width=0.8\textwidth]{bian1.png}
    \end{figure}
    
    \begin{figure}[H]
        \centering
        \includegraphics[width=0.8\textwidth]{bian2.png}
    \end{figure}
    
    \begin{figure}[H]
        \centering
        \includegraphics[width=0.8\textwidth]{bian3.png}
    \end{figure}
    
    \begin{figure}[H]
        \centering
        \includegraphics[width=0.8\textwidth]{bian4.png}
    \end{figure}
    
    \begin{figure}[H]
        \centering
        \includegraphics[width=0.8\textwidth]{bian5.png}
    \end{figure}
    
    \begin{figure}[H]
        \centering
        \includegraphics[width=0.8\textwidth]{bian6.png}
    \end{figure}
    
\end{definition}

\begin{definition}
    \begin{figure}[H]
        \centering
        \includegraphics[width=0.8\textwidth]{bian7.png}
    \end{figure}

    \begin{enumerate}[left=0pt]
        \textbf{\textcolor{red}{Attention}}
        \item 等效对外部(端钮以外)有效,对内不成立。
        \item 等效电路与外部电路无关。
        \item 用于简化电路。
    \end{enumerate}
\end{definition}

\cleardoublepage

\begin{definition}
    \begin{figure}[H]
        \centering
        \includegraphics[width=0.8\textwidth]{bian8.png}
    \end{figure}
\end{definition}

\section{电压源、电流源的串联和并联}

\begin{definition}
    1.理想电压源的串联和并联
    相同的理想电压源才能并联,电源中的电流不确定。

    2.理想电流源的串联并联
    相同的理想电流源才能串联,每个电流源的端电压不能确定。
\end{definition}

\section{实际电源的两种模型及其等效变换}

\cleardoublepage

\begin{definition}
    \begin{figure}
        \centering
        \includegraphics[width=0.8\textwidth]{sj1.png}
    \end{figure}

    \vspace{1mm} % 调整图片之间的垂直间距

    \begin{figure}[H]
        \centering
        \includegraphics[width=0.8\textwidth]{sj2.png}
    \end{figure}

    \vspace{1mm} % 调整图片之间的垂直间距

    \begin{figure}[H]
        \centering
        \includegraphics[width=0.8\textwidth]{sj3.png}
    \end{figure}

    \vspace{1mm} % 调整图片之间的垂直间距

    \begin{figure}[H]
        \centering
        \includegraphics[width=0.8\textwidth]{sj4.png}
    \end{figure}

    \vspace{1mm} % 调整图片之间的垂直间距

    \begin{figure}[H]
        \centering
        \includegraphics[width=0.8\textwidth]{sj5.png}
    \end{figure}
\end{definition}

\section{输入电阻}

\begin{definition}
    \begin{figure}[H]
        \centering
        \includegraphics[width=0.8\textwidth]{zu1.png}
    \end{figure}

    \begin{figure}[H]
        \centering
        \includegraphics[width=0.8\textwidth]{zu2.png}
    \end{figure}
\end{definition}

\chapter{电阻电路的一般分析}

\section{电路的图}

\begin{definition}
    \begin{figure}[H]
        \centering
        \includegraphics[width=0.8\textwidth]{tu1.png}
    \end{figure}

    \begin{figure}[H]
        \centering
        \includegraphics[width=0.8\textwidth]{tu2.png}
    \end{figure}

    \begin{figure}[H]
        \centering
        \includegraphics[width=0.8\textwidth]{tu3.png}
    \end{figure}

    \begin{figure}[H]
        \centering
        \includegraphics[width=0.8\textwidth]{tu4.png}
    \end{figure}

    \begin{figure}[H]
        \centering
        \includegraphics[width=0.8\textwidth]{tu5.png}
    \end{figure}

    \begin{figure}[H]
        \centering
        \includegraphics[width=0.8\textwidth]{tu6.png}
    \end{figure}

    \begin{figure}[H]
        \centering
        \includegraphics[width=0.8\textwidth]{tu7.png}
    \end{figure}

    \begin{figure}[H]
        \centering
        \includegraphics[width=0.8\textwidth]{tu8.png}
    \end{figure}

    \begin{figure}[H]
        \centering
        \includegraphics[width=0.8\textwidth]{tu9.png}
    \end{figure}
\end{definition}

\section{KCL和KVL的独立方程数}

\begin{definition}
    \begin{figure}[H]
        \centering
        \includegraphics[width=0.8\textwidth]{fang1.png}
    \end{figure}

    \begin{figure}[H]
        \centering
        \includegraphics[width=0.8\textwidth]{fang2.png}
    \end{figure}

    \begin{figure}[H]
        \centering
        \includegraphics[width=0.8\textwidth]{fang3.png}
    \end{figure}
\end{definition}


\section{支路电流法}

\begin{definition}
    \begin{figure}[H]
        \centering
        \includegraphics[width=0.8\textwidth]{zl1.png}
    \end{figure}

    \begin{figure}[H]
        \centering
        \includegraphics[width=0.8\textwidth]{zl2.png}
    \end{figure}

    \begin{figure}[H]
        \centering
        \includegraphics[width=0.8\textwidth]{zl3.png}
    \end{figure}

    \begin{figure}[H]
        \centering
        \includegraphics[width=0.8\textwidth]{zl4.png}
    \end{figure}

    \begin{figure}[H]
        \centering
        \includegraphics[width=0.8\textwidth]{zl5.png}
    \end{figure}

    \begin{figure}[H]
        \centering
        \includegraphics[width=0.8\textwidth]{zl6.png}
    \end{figure}

    \begin{figure}[H]
        \centering
        \includegraphics[width=0.8\textwidth]{zl7.png}
    \end{figure}

    \begin{figure}[H]
        \centering
        \includegraphics[width=0.8\textwidth]{zl8.png}
    \end{figure}
\end{definition}


\section{网孔电流法}

\begin{definition}
    \begin{figure}
        \centering
        \includegraphics[width=0.8\textwidth]{wl1.png}
    \end{figure}

    \begin{figure}
        \centering
        \includegraphics[width=0.8\textwidth]{wl2.png}
    \end{figure}

    \begin{figure}
        \centering
        \includegraphics[width=0.8\textwidth]{wl3.png}
    \end{figure}

    \begin{figure}
        \centering
        \includegraphics[width=0.8\textwidth]{wl4.png}
    \end{figure}

    \begin{figure}
        \centering
        \includegraphics[width=0.8\textwidth]{wl5.png}
    \end{figure}

    \begin{figure}
        \centering
        \includegraphics[width=0.8\textwidth]{wl6.png}
    \end{figure}

    \begin{figure}
        \centering
        \includegraphics[width=0.8\textwidth]{wl7.png}
    \end{figure}

    \begin{figure}
        \centering
        \includegraphics[width=0.8\textwidth]{wl8.png}
    \end{figure}
\end{definition}

\section{回路电流法}

\begin{definition}
    \begin{figure}[H]
        \centering
        \includegraphics[width=0.8\textwidth]{hl1.png}
    \end{figure}

    \begin{figure}[H]
        \centering
        \includegraphics[width=0.8\textwidth]{hl2.png}
    \end{figure}

    \begin{figure}[H]
        \centering
        \includegraphics[width=0.8\textwidth]{hl3.png}
    \end{figure}

    \begin{figure}[H]
        \centering
        \includegraphics[width=0.8\textwidth]{hl4.png}
    \end{figure}

    \begin{figure}[H]
        \centering
        \includegraphics[width=0.8\textwidth]{hl5.png}
    \end{figure}

    \begin{figure}[H]
        \centering
        \includegraphics[width=0.8\textwidth]{hl6.png}
    \end{figure}

    \begin{figure}[H]
        \centering
        \includegraphics[width=0.8\textwidth]{hl7.png}
    \end{figure}

    \begin{figure}[H]
        \centering
        \includegraphics[width=0.8\textwidth]{hl8.png}
    \end{figure}
\end{definition}

\section{结点电压法}

\begin{definition}
    1. 结点电压法

    以结点电压为未知量列写电路方程分析电路的方法。适用于结点较少的电路。

    \textbf{基本思想}:
    \begin{itemize}
        \item 选结点电压为未知量,则 KVL 自动满足,无需列写 KVL 方程。各支路电流、电压可视为结点电压的线性组合,求出结点电压后,便可方便地得到各支路电压、电流。
    \end{itemize}

    \textbf{列写的方程}:
    \begin{itemize}
        \item 结点电压法列写的是结点上的 KCL 方程,独立方程数为:$(n - 1)$
        \begin{enumerate}
            \item 与支路电流法相比,方程数减少 $b - (n - 1)$ 个。
            \item 任意选择参考点:其它结点与参考点的电位差即为结点电压(位),方向为从独立结点指向参考结点。
        \end{enumerate}
    \end{itemize}
\end{definition}

\begin{definition}
    \begin{figure}[H]
        \centering
        \includegraphics[width=0.8\textwidth]{jd1.png}
    \end{figure}

    \begin{figure}[H]
        \centering
        \includegraphics[width=0.8\textwidth]{jd2.png}
    \end{figure}

    \begin{figure}[H]
        \centering
        \includegraphics[width=0.8\textwidth]{jd3.png}
    \end{figure}

    \begin{figure}[H]
        \centering
        \includegraphics[width=0.8\textwidth]{jd4.png}
    \end{figure}

    \begin{figure}[H]
        \centering
        \includegraphics[width=0.8\textwidth]{jd5.png}
    \end{figure}

    \begin{figure}[H]
        \centering
        \includegraphics[width=0.8\textwidth]{jd6.png}
    \end{figure}

    \begin{figure}[H]
        \centering
        \includegraphics[width=0.8\textwidth]{jd7.png}
    \end{figure}

    \begin{figure}[H]
        \centering
        \includegraphics[width=0.8\textwidth]{jd8.png}
    \end{figure}

    \begin{figure}[H]
        \centering
        \includegraphics[width=0.8\textwidth]{jd9.png}
    \end{figure}

    \begin{figure}[H]
        \centering
        \includegraphics[width=0.8\textwidth]{jd10.png}
    \end{figure}

    \begin{figure}[H]
        \centering
        \includegraphics[width=0.8\textwidth]{jd11.png}
    \end{figure}

    \begin{figure}[H]
        \centering
        \includegraphics[width=0.8\textwidth]{jd12.png}
    \end{figure}

    \begin{figure}[H]
        \centering
        \includegraphics[width=0.8\textwidth]{jd13.png}
    \end{figure}
\end{definition}


\chapter{电路定理}

\section{叠加定理}

\begin{definition}
    1. 叠加定理

    在线性电路中,任一支路的电流(或电压)可以看成是电路中每一个独立电源单独作用于电路时,在该支路产生的电流(或电压)的代数和。

    \begin{figure}[H]
        \centering
        \includegraphics[width=0.8\textwidth]{dj1.png}
    \end{figure}

    \begin{figure}[H]
        \centering
        \includegraphics[width=0.8\textwidth]{dj2.png}
    \end{figure}

    \textbf{结论}:
    结点电压和支路电流均为各电源的一次函数,均可看成各独立电源单独作用时产生的响应之叠加。

    \textbf{几点说明}:
    \begin{enumerate}
        \item 叠加定理只适用于线性电路。
        \item 一个电源作用,其余电源为零。
        \begin{itemize}
            \item 电压源为零 — 短路。
            \item 电流源为零 — 开路。
        \end{itemize}
    \end{enumerate}

    \begin{figure}[H]
        \centering
        \includegraphics[width=0.8\textwidth]{dj3.png}
    \end{figure}

    \begin{figure}[H]
        \centering
        \includegraphics[width=0.8\textwidth]{dj4.png}
    \end{figure}

    \begin{figure}[H]
        \centering
        \includegraphics[width=0.8\textwidth]{dj5.png}
    \end{figure}

    \begin{figure}[H]
        \centering
        \includegraphics[width=0.8\textwidth]{dj6.png}
    \end{figure}

    \begin{figure}[H]
        \centering
        \includegraphics[width=0.8\textwidth]{dj7.png}
    \end{figure}

    \begin{figure}[H]
        \centering
        \includegraphics[width=0.8\textwidth]{dj8.png}
    \end{figure}

    \begin{figure}[H]
        \centering
        \includegraphics[width=0.8\textwidth]{dj9.png}
    \end{figure}
\end{definition}

\cleardoublepage

\section{替代定理}

\begin{definition}
    1. 替代定理

    对于给定的任意一个电路,若某一支路电压为 $u_k$、电流为 $i_k$,那么这条支路就可以用一个电压等于 $u_k$ 的独立电压源,或者用一个电流等于 $i_k$ 的独立电流源,或用 $R = \frac{u_k}{i_k}$ 的电阻来替代,替代后电路中全部电压和电流均保持原有值(解答唯一)。

    \begin{figure}[H]
        \centering
        \includegraphics[width=0.8\textwidth]{ti1.png}
    \end{figure}

    \begin{figure}[H]
        \centering
        \includegraphics[width=0.8\textwidth]{ti2.png}
    \end{figure}

    \begin{figure}[H]
        \centering
        \includegraphics[width=0.8\textwidth]{ti3.png}
    \end{figure}

    \textbf{原因}:\par
    替代前后 KCL、KVL 关系相同,其余支路的 $u$、$i$ 关系不变。用 $u_k$ 替代后,其余支路电压不变(KVL),其余支路电流也不变,故第 $k$ 条支路 $i_k$ 也不变(KCL)。用 $i_k$ 替代后,其余支路电流不变(KCL),其余支路电压不变,故第 $k$ 条支路 $u_k$ 也不变(KVL)。

    \textbf{注意}:
    \begin{enumerate}
        \item 替代定理既适用于线性电路,也适用于非线性电路。
    \end{enumerate}

    \begin{figure}[H]
        \centering
        \includegraphics[width=0.8\textwidth]{ti4.png}
    \end{figure}

    \begin{figure}[H]
        \centering
        \includegraphics[width=0.8\textwidth]{ti5.png}
    \end{figure}

    \begin{figure}[H]
        \centering
        \includegraphics[width=0.8\textwidth]{ti6.png}
    \end{figure}

    \begin{figure}[H]
        \centering
        \includegraphics[width=0.8\textwidth]{ti7.png}
    \end{figure}

    \begin{figure}[H]
        \centering
        \includegraphics[width=0.8\textwidth]{ti8.png}
    \end{figure}
\end{definition}


\section{戴维宁定理和诺顿定理}

\begin{definition}
    \foreach \i in {1,...,13} {
    \begin{figure}[H]
        \centering
        \includegraphics[width=1\textwidth]{dai\i.png}
    \end{figure}
}
\end{definition}


\section{最大功率传输定理}

\begin{definition}
    \foreach \i in {1,...,4} {
    \begin{figure}[H]
        \centering
        \includegraphics[width=1\textwidth]{zui\i.png}
    \end{figure}
}
\end{definition}

\section{特勒根定理}

\begin{definition}
    \foreach \i in {1,...,10} {
    \begin{figure}[H]
        \centering
        \includegraphics[width=1\textwidth]{te\i.png}
    \end{figure}
}
\end{definition}


\section{互易定理}

\begin{definition}
    互易定理

    互易性是一类特殊的线性网络的重要性质。一个具有互易性的网络在输入端(激励)与输出端(响应)互换位置后,同一激励所产生的响应并不改变。具有互易性的网络叫互易网络,互易定理是对电路的这种性质所进行的概括,它广泛的应用于网络的灵敏度分析和测量技术等方面。

    \textbf{1. 互易定理}:

    对一个仅含电阻的二端口电路 NR,其中一个端口加激励源,一个端口作响应端口,在只有一个激励源的情况下,当激励与响应互换位置时,同一激励所产生的响应相同。
\end{definition}

\begin{figure}[H]
    \centering
    \includegraphics[width=1\textwidth]{hu1.png}
\end{figure}

\begin{figure}[H]
    \centering
    \includegraphics[width=1\textwidth]{hu2.png}
\end{figure}

\begin{figure}[H]
    \centering
    \includegraphics[width=1\textwidth]{hu3.png}
\end{figure}

\begin{figure}[H]
    \centering
    \includegraphics[width=1\textwidth]{hu4.png}
\end{figure}

\begin{figure}[H]
    \centering
    \includegraphics[width=1\textwidth]{hu5.png}
\end{figure}

\textbf{应用互易定理分析电路时应注意}:
\begin{enumerate}
    \item 互易前后应保持网络的拓扑结构不变,仅理想电源搬移;
    \item 互易前后端口处的激励和响应按照关联参考方向推导的;
    \item 互易定理只适用于线性电阻网络在单一电源激励下,端口两个支路电压电流关系。
\end{enumerate}

\begin{figure}[H]
    \centering
    \includegraphics[width=1\textwidth]{hu6.png}
\end{figure}

\begin{figure}[H]
    \centering
    \includegraphics[width=1\textwidth]{hu7.png}
\end{figure}

\begin{figure}[H]
    \centering
    \includegraphics[width=1\textwidth]{hu8.png}
\end{figure}

\begin{figure}[H]
    \centering
    \includegraphics[width=1\textwidth]{hu9.png}
\end{figure}

\begin{figure}[H]
    \centering
    \includegraphics[width=1\textwidth]{hu10.png}
\end{figure}



\section{储能元件(第六章)}

\subsection{电容元件}

\begin{definition}
    \foreach \i in {1,...,18} {
    \begin{figure}[H]
        \centering
        \includegraphics[width=1\textwidth]{rong\i.png}
    \end{figure}
}
\end{definition}




\subsection{电感元件}


\begin{definition}
    \foreach \i in {1,...,11} {
    \begin{figure}[H]
        \centering
        \includegraphics[width=1\textwidth]{gan\i.png}
    \end{figure}
}
\end{definition}

\subsection{电容、电感元件的串联与并联}

\begin{definition}
    \foreach \i in {1,...,9} {
    \begin{figure}[H]
        \centering
        \includegraphics[width=1\textwidth]{rg\i.png}
    \end{figure}
}
\end{definition}

\begin{definition}
    以上虽然是关于两个电容或两个电感的串联和并联等效,但其结论可以推广到 n 个电容或 n 个电感的串联和并联等效。
\end{definition}



\chapter{一阶电路的时域分析}

\section{动态电路的方程及其初始条件}

1.动态电路
含有动态元件电容和电感的电路称动态电路。
特点
当动态电路状态发生改变时(换路)需要经历一个变化过程才
能达到新的稳定状态。这个变化过程称为电路的过渡过程。

\begin{definition}
    \foreach \i in {1,...,23} {
    \begin{figure}[H]
        \centering
        \includegraphics[width=1\textwidth]{dong\i.png}
    \end{figure}
}
\end{definition}

\section{一阶电路的零输入响应}

\begin{definition}
    \foreach \i in {1,...,18} {
    \begin{figure}[H]
        \centering
        \includegraphics[width=1\textwidth]{ling\i.png}
    \end{figure}
}
\end{definition}

\end{document}